%%%%%%%%%%%%%%%%%%%%%%%%%%%%%%%%%%%%%%%%%%%%%%%%%%%%%%%%%%%%%%%%%%%%%%%%%
% All the content is in one file because I do not expect multiple editors.
%%%%%%%%%%%%%%%%%%%%%%%%%%%%%%%%%%%%%%%%%%%%%%%%%%%%%%%%%%%%%%%%%%%%%%%%%

\documentclass[11pt]{article}

%%%%%%%%%%%%%%%%%%%%%%%%%%%%%%%%%%%%%%%%%%%%%%%%%%%%%%%%%%%%%%%%%%%%%%%%%
%% packages

\usepackage{latexsym}
\usepackage{algorithm}
\usepackage{algorithmic}
\usepackage{graphicx}
\usepackage{subfigure}
\usepackage[T1]{fontenc}
% \usepackage{mathptmx}
\usepackage{newcent}
%\usepackage{times}
\usepackage{amsmath}
\usepackage{amssymb}
\usepackage{amsfonts}
\usepackage{fullpage}
% \usepackage{complexity}
\usepackage{hyphenat}
\usepackage{multirow}

%\usepackage[lite, subscriptcorrection]{mtpro2}

\usepackage[mathscr]{euscript}

\renewcommand{\P}{\mathbb{P}}
\newcommand{\E}{\mathbb{E}}
\newcommand{\Q}{\mathbb{Q}}
\newcommand{\R}{\mathbb{R}}
\newcommand{\Z}{\mathbb{Z}}
\newcommand{\N}{\mathbb{N}}
\newcommand{\C}{\mathbb{C}}
\newcommand{\K}{\mathbb{K}}
\newcommand{\cA}{\mathscr A}
\newcommand{\cF}{\mathcal F}
\newcommand{\cB}{\mathscr B}
\newcommand{\cM}{\mathscr M}
\newcommand{\cG}{\mathscr G}
\newcommand{\cP}{\mathscr P}
\newcommand{\cL}{\mathscr L}
\newcommand{\cX}{\mathscr X}
\newcommand{\cZ}{\mathscr Z}
\newcommand{\cE}{\mathscr E}
\newcommand{\cN}{\mathscr N}
\newcommand{\cT}{\mathscr T}
\newcommand{\ran}{\text{ran}}
\newcommand{\dom}{\text{dom}}
\newcommand{\supp}{\text{supp}}
\newcommand{\eps}{\varepsilon}
\newcommand{\var}{\text{Var}}
\newcommand{\ind}{{\mathbf 1}}

%%%%%%%%%%%%%%%%%%%%%%%%%%%%%%%%%%%%%%%%%%%%%%%%%%%%%%%%%%%%%%%%%%%%%%%%%
%% basic definitions, environments

\newtheorem{theorem}{Theorem}
\newtheorem{lemma}{Lemma}
\newtheorem{corollary}[theorem]{Corollary}
\newtheorem{definition}{Definition}
\newtheorem{property}{Property}
\newtheorem{observation}{Observation}
\newtheorem{remark}{Remark}

\newenvironment{proof}
        {\noindent {\em Proof.}~~~} %\\
        {\begin{flushright}$\Box$\end{flushright}}

\addtolength{\oddsidemargin}{-0.25in}
\addtolength{\evensidemargin}{-0.25in}
\addtolength{\textwidth}{0.5in}
\addtolength{\topmargin}{-.25in}
\addtolength{\textheight}{0.75in}	

%%%%%%%%%%%%%%%%%%%%%%%%%%%%%%%%%%%%%%%%%%%%%%%%%%%%%%%%%%%%%%%%%%%%%%%%%
%% title details

\title{
  Written Assignment 1 - Solutions\\
  \large
  Vancouver Summer Program 2019 -- Algorithms -- UBC \\
  \vspace*{0.2in} \hrule
}

%\author{
%	Sathish Gopalakrishnan
%}

\date{}

%%%%%%%%%%%%%%%%%%%%%%%%%%%%%%%%%%%%%%%%%%%%%%%%%%%%%%%%%%%%%%%%%%%%%%%%%

\begin{document}

\maketitle

\setlength{\baselineskip}{0.90\baselineskip}

%%%%%%%%%%%%%%%%%%%%%%%%%%%%%%%%%%%%%%%%%%%%%%%%%%%%%%%%%%%%%%%%%%%%%%%%%
%% the abstract

%% no abstract needed

%%%%%%%%%%%%%%%%%%%%%%%%%%%%%%%%%%%%%%%%%%%%%%%%%%%%%%%%%%%%%%%%%%%%%%%%%

\pagestyle{empty}

\vspace*{-0.75in}

 \begin{enumerate}
\item (Enter Fibonacci) The Fibonacci sequence is defined as follows: $F_{0} = F_{1} = 1$, and $F_{n} = F_{n-1} + F_{n-2}$ for all integers $n \geq 2$.
  \begin{enumerate}
%    \item Use mathematical induction to show that $F_{n} \geq 2^{n/2}$ for all $n \geq 6$. There is more than one base case; identify all base cases!
      \item You are to derive an efficient algorithm to compute the $n$th Fibonacci number. Observe that 
\begin{align*}
  F_{n} & = F_{n-1} + F_{n-2}\\
  F_{n-1} & = F_{n-1} + 0 \cdot F_{n-2}.
\end{align*}

If we write this linear system in terms of matrices, we have
\begin{align*}
  \begin{bmatrix}
    F_{n} \\ 
  F_{n-1}
  \end{bmatrix} =
  \begin{bmatrix}
    1&1\\
    1 & 0
  \end{bmatrix}
\begin{bmatrix}
    F_{n-1} \\ 
  F_{n-2}
\end{bmatrix}
\end{align*}

Using this linear relation, derive an algorithm to compute $F_{n}$. Your algorithm should run in time $O (\log n)$. \\\ \textbf{Hint:} Use repeated squaring to compute matrix powers.\\

\textbf{Solution.} Unfolding the given linear recurrence, we have  
\begin{align*}
  \begin{bmatrix}
    F_{n-1} \\ 
  F_{n-2}
\end{bmatrix} = \begin{bmatrix}
    1&1\\
    1 & 0
  \end{bmatrix}
\begin{bmatrix}
    F_{n-2} \\ 
  F_{n-3}
\end{bmatrix},
\end{align*}
which gives us 
\begin{align*}
  \begin{bmatrix}
    F_{n} \\ 
  F_{n-1}
  \end{bmatrix} =
  \begin{bmatrix}
    1&1\\
    1 & 0
  \end{bmatrix}^{2}
\begin{bmatrix}
    F_{n-2} \\ 
  F_{n-3}
\end{bmatrix}.
\end{align*}

Continuing in this manner recursively, we get, for every $n \geq 1$,
\begin{align*}
  \begin{bmatrix}
    F_{n} \\ 
  F_{n-1}
  \end{bmatrix} =
  \begin{bmatrix}
    1&1\\
    1 & 0
  \end{bmatrix}^{n-1}
\begin{bmatrix}
    F_{1} \\ 
  F_{0}
\end{bmatrix} = \begin{bmatrix}
    1&1\\
    1 & 0
  \end{bmatrix}^{n-1}
\begin{bmatrix}
    1 \\ 
  1
\end{bmatrix},
\end{align*}
so $F_{n}$ is the top entry of the vector $\begin{bmatrix}
    1&1\\
    1 & 0
  \end{bmatrix}^{n-1}
\begin{bmatrix}
    1 \\ 
  1
\end{bmatrix}$,
with the understanding that for any matrix $A$, $A^{0} = I$, the identity matrix.
Then for a given $n \geq 1$, one needs to compute the $(n-1)$st power of $\begin{bmatrix}
  1&1\\
  1 & 0
\end{bmatrix}$. This can be done using repeated squaring with $\Theta(\log_{2}n)$ matrix multiplications, and since our matrices are $2 \times 2$ and are thus of \emph{constant} size (relative to the input $n$), multiplying two matrices requires a constant number (in $n$) of integer additions and multiplications. Thus computing $F_{n}$ requires $\Theta(\log n)$ additions/multiplications, which is indeed polynomial in the size of the input, $\log_{2}n$.


  \item Now suppose that writing every bit of the output to memory counts as an operation that we wish to account for in our running-time analysis (in the previous part, we disregard the time required to write the output to memory). Can you compute $F_{n}$ in time that is bounded by a polynomial in the size of the input? Justify your answer.\\

 \textbf{Solution.} We can show by induction that $F_{n} \geq 2^{n/2}$ for all $n \geq 6$. Thus the number of bits required to encode the output in binary is at least $\log_{2}2^{n/2} = n/2$. Since writing each bit to memory counts as an operation here, we will need at least $n/2$ operations to write the entirety of $F_{n}$. Thus $T(n) \geq n/2$. The input is an integer $n \geq 0$ so the size of the input is $\texttt{size}(n) = \log_{2}n$ bits. But $n = 2^{\log_{2}(n)} = 2^{\texttt{size}(n)}$, which is an exponential function in $\texttt{size}(n)$. Thus $T(n) \geq {1 \over 2} 2^{\texttt{size}(n)}$, so $T(n)$ cannot be a polynomial in $\texttt{size}(n)$.


\item (\textbf{Bonus}) Find $a$ if $a$ and $b$ are integers such that $x^2 - x - 1$ is a factor of $ax^{17} + bx^{16} + 1$. \textbf{Hint}: The answer is $F_{n}$ for some $n \geq 1$. It is enough to show this and find $n$ explicitly; you do not need to compute $F_{n}$. \\

\textbf{Solution.} Here is one possible solution. Let's work backwards! Let $F(x) = ax^{17} + bx^{16} + 1$ and let $P(x)$ be the polynomial such that $P(x)(x^2 - x - 1) = F(x)$.

Clearly, the constant term of $P(x)$ must be $- 1$. Now, we have $(x^2 - x - 1)(c_1x^{15} + c_2x^{14} + \cdots + c_{15}x - 1)$, where $c_{i}$ is some coefficient. However, since $F(x)$ has no $x$ term, it must be true that $c_{15} = 1$.

Let's find $c_{14}$ now. Notice that all we care about in finding $c_{14}$ is that $(x^2 - x - 1)(\cdots + c_{14}x^2 + x - 1) = \text{something} + 0x^2 + \text{something}$. Therefore, $c_{14} = - 2$. Undergoing a similar process, $c_{13} = 3$, $c_{12} = - 5$, $c_{11} = 8$, and we see a nice pattern. The coefficients of $P(x)$ are just the Fibonacci sequence with alternating signs! Therefore, $a = c_1 = F_{16}$, where $F_{16}$ denotes the 16th Fibonnaci number and $a = 987$. 

  \end{enumerate}
% \item \textbf{A Gentle Mathematical Warm-up.} 
%   Here we will show that for every positive integer $n$, it is necessary
%   for $2^n + 1$ to be a prime number that $n$ is a \emph{power of two}. In
%     other words, we will prove the statement ``if $n$ a positive integer and
%     $2^n + 1$ is prime, then $n$ is a power of two.''
%     \begin{enumerate}
%     \item Show that for any reals $a$ and $b$, $a \neq b$, and positive integer $n$,
%       $(a-b) | (a^n - b^n)$. Do \emph{not} use induction! \textbf{Hint}: Use
%       geometric series to expand $a^n - b^n$.
%     \item Deduce that if $n$ is a positive integer and $2^{n}+1$ is prime,
%       then $n = 2^{k}$ for some $k\geq 0$. \textbf{Hint:} Assume
%       that $n$ is not a power of two but $2^{n} + 1$ is prime, and use the
%       previous part to arrive at a contradiction.
%     \item Accordingly, is $2^{50} + 1$ a prime number? Can you conclude using
%       the previous part solely whether or not $257$ is prime? Justify your
%       answers.
%     \end{enumerate}

    
% \item \textbf{Enter Fibonacci.} The Fibonacci sequence is defined as follows: $F_{0} = F_{1} = 1$, and $F_{n} = F_{n-1} + F_{n-2}$ for all integers $n \geq 2$.
%   \begin{enumerate}    
%   \item Use mathematical induction to show that $F_{n} \geq 2^{n/2}$ for all $n \geq 6$. There is more than one base case; identify all base cases! 

% %https://artofproblemsolving.com/wiki/index.php?title=1988_AIME_Problems/Problem_13
% \item (\textbf{Bonus}) Find $a$ if $a$ and $b$ are integers such that $x^2 - x - 1$ is a factor of $ax^{17} + bx^{16} + 1$. \textbf{Hint}: The answer is $F_{n}$ for some $n \geq 1$. It is enough to show this and find $n$ explicitly; you do not need to compute $F_{n}$.  
%   \end{enumerate}

% \item (Studyitis) During VSP 2018, the students are sitting in an $n \times n$ grid. A sudden outbreak of studyitis (a rare condition that lasts forever; symptoms include yearning for homework) causes some students to get infected. The infection begins to spread every minute (in discrete time-steps). Two students are considered adjacent if they share an edge (i.e., front, back, left or right, but NOT diagonal); thus, each student is adjacent to 2, 3 or 4 others. A student is infected in the next time step if either
%   \begin{itemize}
%   \item the student was previously infected (there is no cure for this condition), or
%   \item the student is adjacent to {\em at least two} students who are affected by studyitis.
%   \end{itemize}
  
%   Prove that if less than $n$ students are initially infected then the whole class will not be completely infected.
  
%   \textbf{Hint:} Can you identify some property at the initial stage and prove, by induction, that this property is preserved at each time step? Think of {\em invariants}.

\item (Time Complexity)

\begin{enumerate} 
\item Algorithms $A$ and $B$ spend exactly $T_{A}(n) = 0.1n^{2} \log_{10} (n)$ and $T_{B}(n) = 2.5n^{2}$ microseconds, respectively, for a problem of size $n$. Choose the algorithm, which is better in the Big-Oh sense, and find out a problem size $n_{0}$ such that for any larger size $n > n_{0}$ the chosen algorithm outperforms the other. If your problems are of the size $n \leq 10^{9}$, which algorithm will you recommend to use?\\

\textbf{Solution.}
In the Big-Oh sense, algorithm \textbf{B} is better. It outperforms algorithm \textbf{A} when $T_{\textbf{B}}(n) \leq T_{\textbf{A}}(n)$, that is, when $2.5 n^{2} \leq 0.1 n^{2} \log_{10} n$. This inequality reduces to $\log_{10} n \geq 25$, or $n \geq n_{0} = 10^{25}$. If $n \leq 10^{9}$, the algorithm of choice is \textbf{A}.

\item Let $f(n) = (\log n)^{\log n}$ and $g(n) = 2^{(\log_{2} n)^{2}}$. Determine whether $f \in O(g)$, $f \in \Omega(g)$, or both (in which case $f \in \Theta(g)$).\\

\textbf{Solution.} $g(n) = 2^{(\log_{2} n)^{2}} = 2^{(\log_{2} n)(\log_{2} n)} = 
\left(2^{\log_{2} n}\right)^{\log_{2}n} = n^{\log_{2} n}$. Thus, $f \in O(g)$ (in fact, $f \in o(g)$), but $f \notin \Omega(g)$.

\item Show that for any $f,g: \Z_{+} \to \R_{+}$, $O(f+g) = O\bigl(\max\{f,g\}\bigr)$. Recall that $O(\cdot)$ is a set (see notes \#1), and therefore one has to show both $O(f+g) \subset O\bigl(\max\{f,g\}\bigr )$ and $O\bigl (\max\{f,g\} \bigr ) \subset O(f+g)$. \\

\textbf{Solution.} If $h \in  O(f+g)$, then there is $c > 0$ and $N \in \N$ such that $h(n) \leq c\bigl(g(n) + f(n)\bigr)$ for all $n \geq N$. But 
\begin{align*}
  \max\{f,g\} = \frac{f+g + |f-g|}{2},
\end{align*}
from which it follows that $f+g \leq 2 \max\{f,g\}-|f-g| \leq 2 \max\{f,g\}$. Thus we may take $c'=2c$ and $N'=N$ so that $h(n) \leq c' \max\{g(n), f(n)\}$ for all $n \geq N'$. For the other direction, if $h \in O(\max\{f,g\})$, then there is $c > 0$ and $N \in \N$ for which $h(n) \leq c \max\{f(n),g(n)\}$. Then $h(n) \leq c f(n)$ and $h(n) \leq c g(n)$ for all $n \geq N$. Thus $2h(n) \leq c \bigl ( f(n) + g(n) \bigr)$ for all $n \geq N$, so we may take $c' = c/2$ and $N'=n$.
% \item Find the running time of the following procedure in Big-Oh notation.

% \begin{empheq}[box=\fbox]{align*}
%     &\underline{\textsc{Foo}(N)}\\
%     & k \leftarrow 0\\
%     & \text{for } i \leftarrow 1 \text{ to } N\\
%     & \qquad \text{for } j \leftarrow 1 \text{ to } N\\
%     & \qquad \qquad k \leftarrow i+j\\
%     & \text{return } k 
% \end{empheq}


  \end{enumerate}

%\item (Testing i*Phones.) You are testing the physical durability of the new i*Phones to determine the (maximum) height from which they can fall and still not break. You are performing this testing in increments of half a feet. You want to find the largest height, $X$, from which an i*Phone can be safely dropped. If you wanted to minimize the number of devices used for this test, you could start at 0.5 feet, and keep increasing the height to a maximum of $H \times 0.5$ feet, until a drop breaks the device. This sequential test would require only one device for testing but would require (potentially) a large number of tests. There is a natural tradeoff between the number of devices you break during the testing and number of tests performed. (For example, a binary search between $0$ and $H$ would require $\log{H}$ devices in the worst case.) 
%	
%	Suppose your manager at a*Pple Corp. limits you to a testing budget of $n$ devices, determine an efficient strategy for identifying $X$. Let $g_{n}(H)$ be the number of tests needed by a strategy given a maximum height $H$ and $n$ devices. A strategy is efficient if $\lim_{H \to \infty} g_{n}(H)/g_{n-1}(H) = 0$.
%	
%	{\em Clearly describe your testing strategy and prove that it is indeed efficient.}


\end{enumerate}



%%%%%%%%%%%%%%%%%%%%%%%%%%%%%%%%%%%%%%%%%%%%%%%%%%%%%%%%%%%%%%%%%%%%%%%%%
%% the bibliography starts here.

%\newpage
%\bibliographystyle{acm}
%\setlength{\baselineskip}{0.8\baselineskip}
%\bibliography{../../Bibliography/CompleteBibliography}

%%%%%%%%%%%%%%%%%%%%%%%%%%%%%%%%%%%%%%%%%%%%%%%%%%%%%%%%%%%%%%%%%%%%%%%%%
%% other closing material

% \appendix
% \input{appendix.tex}

\end{document}
%%% Local Variables:
%%% mode: latex
%%% TeX-master: t
%%% End:
